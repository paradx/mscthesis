\begin{abstract}
\noindent
Urban Heat Islands (\acp{UHI}) pose a growing health risk by exacerbating heat stress for residents of urban areas. 
Due to the increased prevalence of extreme weather events and heat waves and as a cause for higher energy consumption, \acp{UHI} become more relevant as a topic for city planners and policy makers to consider.
Identification of areas most impacted or at risk require a data backed tool set to aid urban planning. 
This study presents a comprehensive pipeline developed using Python to automate the processing and analysis of Landsat 7,8 and 9 remote sensing imagery.
The pipeline facilitates the generation of Normalized Difference Vegetation Index (NDVI) and heat maps, running statistical analysis of factors known to create \acp{UHI} of a provided area of interest, serving as a robust toolset for the investigation and detection of \acp{UHI}.
The methodology employed leverages the spectral characteristics of Landsat data to provide high-resolution insights into temperature variations within urban areas as well as statistical analysis of the composition of the identified Urban Heat Islands.
The ultimate aim is to offer actionable guidance to city planners and developers for the mitigation of \acp{UHI}, by classification of \acp{UHI} on different scales.
At the current implementation level the pipeline is able to detect \acp{UHI} from level one brightness temperature and the statistical analysis indicate a strong correlation between land cover types and heat island intensity, affirming the utility of the pipeline in urban climate studies.
%To analyse and identify Urban Heat Islands using remote sensing imagery
\end{abstract}
