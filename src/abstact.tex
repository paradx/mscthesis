%\begin{abstract}
%\glspl{UHI} pose a growing health risk by exacerbating heat stress for residents of urban areas.
%With the increased prevalence of extreme weather events, urbanization, and heat waves, and as a factor in higher energy consumption, \glspl{UHI} are becoming more relevant for city planners and policymakers.
%This work investigates the impact and influence of rising temperatures and land use change in cities on the severity of the \glspl{UHI}.
%Thermal Infrared Satellite Data is used to measure \glspl{SUHI} effects with machine learning algorithms for creating time series of land use and land cover (LULC) in selected cities, longitudinal measurement of UHI intensity, and the future modelling of \glspl{UHI} based on these findings.
%Additionally, various indices are employed to measure heat stress and other parameters related to UHI intensity.
%This comprehensive approach aims to provide a data-backed toolkit to aid urban planning and decision-making in mitigating UHI effects.
%The results of this study (TODO: ADD RESULTS) will offer insights into the direct and indirect impacts of urbanization and climatic changes on UHI phenomena.
%\end{abstract}

% OLD 
\begin{abstract}
  \todo[inline]{Add überschrift}
\noindent
\glspl{UHI} pose a growing health risk by exacerbating heat stress for residents of urban areas. 
Due to the increased prevalence of extreme weather events, urbanization and heat waves and as a cause for higher energy consumption, \glspl{UHI} become more relevant as a topic for city planners and policy makers to consider.
Identification of areas most impacted or at risk require a data backed tool set to aid urban planning. 
This work investigates the impact and influence of rising temperatures and land use changes in cities on the severity of the \glspl{UHI} within them. 
In this case study different methodologies where combined and extended to find \glspl{UHI} within the city of Bremen. 
The introduction of a statistical measure reduces seasonal effects on severity of \glspl{UHI}.
The study shows that sealed surface types are dominant in strong \glspl{UHI} (with more than 3$\sigma$ above average temperature of the surrounding rural area).
The data indicate an increase in size and a combining of surface urban heat islands with rising temperatures. 
More research could give insight in how this impacts air temperatures and if vegetation corridors within affected areas could limit and mitigate this effect.
\end{abstract}
