 
\subsection{Outlook and Future Work}\label{sec:future}
The overall results of this work allowed to identify many further areas of studies.
The classification part could be improved by using a widely available product that has a suitable resolution. 
During development a 300 m and 10 m reference product were used (see~\cref{sec:references}), the first was considered not precise enough for these measurements.
The second reference product used had 10-m resolution, but was limited by the available temporal resolution of one year, with yearly averages.
% Areas of interest
Since the selected \gls{AoI} did not contain sufficiently large \gls{LULC} changes within the urban area, multiple sides where selected that could be of interest for future studies using the same methodology.
\textbf{Leipzig 416} is a construction project within the city center of Leipzig.
The construction started in 2021 and a significant land use change was observed in that area. \\
One interesting area of study would be the Leipzig 416 Project, an urban district development project, currently being constructed in the city centre of Leipzig, designed to be environmentally friendly, vegetation rich and designed after sponge city concepts.
Monitoring the development of \glspl{UHI} in the area and surrounding could show how effective modern build style is compared to the previous situation.\\
\\
Another interesting non-urban area of study could be large construction projects in previously rural areas, like the Tesla Factory in Grünheide, the Berlin Brandenburg Airport in Germany, the ITER (International Thermonuclear Experimental Reactor) near Marseilles in France and the impact on the surrounding micro climate and fauna caused by the \gls{LULC} change.
\\
Another topic of interest could be the scientific study and monitoring of implemented mitigating techniques.
Allowing rating of effectiveness for installed countermeasures to find optimized, cost effective solutions for the mitigation of \glspl{UHI}, adapted to the local conditions.

