\subsection{Land Surface Temperature Calculation}
\subsubsection{Calculation of the Landsat L2 LST}
The \ac{USGS} provides \ac{LST} data for Landsat 7,8 and 9 Thermal instruments\cite{EROASC2013}\cite{EROASC1999} online after around two weeks after data acquisition. 
The level two data set uses the top of the atmosphere temperature and calculates the \ac{LST} using atmospheric correction, correction for absorption and emissivity based on ASTER GED data set for radiation temperature.
%
\subsubsection{Estimation of the LST temperature based on Landsat L1 Data}\label{sec:lstcalc}
To check the functionality of the pipeline and create a possibility to create land surface temperature maps from raw data a estimation algorithm was implemented based on the Landsat level 1 data. It was correction for emissivity based on the \ac{NDVI}, this is not as accurate as the ASTER data but provides LST maps quickly and can be derived from a single image data set of the Landsat mission. 
The landsat intensity value can be converted to a  brightness temperature using \cref{equ:toa}(see \cite{usgsLSTLandsat})
\begin{equation}
    \begin{split}\label{equ:toa}
      L_{\lambda} &= M_L\cdot Q_{cal} + A_L \\
      BT_{TOA} &= \frac{K2}{\ln(\frac{K1}{L_{\lambda}} + 1)}
    \end{split}
\end{equation}
%
For the $L_{\lambda}$ the \ac{TOA} radiance is calculated from the $M_L$ parameter from the band meta data file and the additive rescaling factor that is also included in the meta data of the image. 
Where K1 and K2 are satellite dependent correction parameters and $L_{\lambda}$ is the observed brightness temperature at the satellite instrument. 
The resulting \ac{TOA} brightness temperature is then corrected using fixed factors based on the \ac{NDVI}. 
The conversion to \ac{LST} can be done using satellite dependent steps and the correction based on \ac{NDVI} are for Landsat 8  and for Landsat 7. %TODO shown in \cref{equ:lstL8}.
\begin{equation}\label{equ:lstL8}
  LST = \frac{T_{TOA}}{1+\frac{\epsilon\cdot T_{TOA}}{1.4388}}
\end{equation}
%
To approximate a proper emissivity correction the \cref{equ:emslstL8}\cite{Sobrino2004} is used.
\begin{equation}
\begin{split}\label{equ:emslstL8}
  \epsilon &= m \cdot P_v + n \\
  P_v &= \frac{NDVI-NDVI_{min}}{NDVI_{max}- NDVI_{min}}
\end{split}
\end{equation}
With $P_v$ beeing the vegetative fraction 
where $n$ and $m$ are correction factor for emissivity of partly vegetated areas.  
%
For \ac{NDVI} values below 0.2 the surface is assumed to be not populated by any green vegetation and the emissivity is taken from the red band\cite{Nichol1994}.
For \ac{NDVI} $>$ 0.5 the area of a pixel is assumed to be fully vegetation and the emmissivity can be assumed as 99\%\cite{Sobrino2004} and the values in between can be calculated by the above method with m and n taken from previous research as $m = 0.004$ and $n = 0.986$
~\cite[equ.~12a\&b]{Sobrino2004}.
%
In the following analysis the $BT_{TOA}$ is used without emissivity correction, to test and show that the raw data can be used to detect \acp{UHI}. 
%The atmospheric influence and the material properties might 

